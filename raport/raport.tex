\documentclass[12pt]{article}

\usepackage[utf8]{inputenc} 
\usepackage[T1]{fontenc}   
\usepackage{hyperref}       
\usepackage{url}           
\usepackage{booktabs}       
\usepackage{amsfonts}       
\usepackage{amsmath}  
\usepackage{nicefrac}      
\usepackage{microtype}      
\usepackage{lipsum}
\usepackage{graphicx}
\usepackage{float}
\usepackage{indentfirst}
\usepackage{setspace}
\usepackage{geometry}
\usepackage{caption}
\usepackage{subcaption}
\usepackage{listings}
\usepackage{pgfplots}
\usepackage{adjustbox}
\geometry{
    a4paper,
    left=25mm,
    right=25mm,
    top=25mm,
    bottom=25mm
}

\begin{document}
\author{Jędrzej Szor 239716}
\title{\text{Obliczenia ewolucyjne}\\\textbf{Implementacja algorytmu PSO}}
\maketitle

\section{Opis zastosowanych współczynników i propozycji}
\subsection{Współczynniki stałe}
Przedstawione wartości współczynników zostały zaproponowane w oparciu o artykuł \cite{parameters}.
\begin{itemize}
    \item $w = 0.3,\quad c1, c2 = 1$,
    \item $w = 0.3,\quad c1, c2 = 2$,
    \item $w = 0.6,\quad c1, c2 = 1$,
    \item $w = 0.6,\quad c1, c2 = 2$
\end{itemize}
\subsection{Współczynniki zmienne (TVAC)}
Współczynniki zmienne zmniejszają się liniowo pomiędzy wartością maksymalną i minimalną w różnych kierunkach dla różnych parametrów \cite{tvac}. Powoduje to bardziej globalny charakter przeszukiwania na początku biegu algorytmu i lokalny charakter pod koniec. Ze względu na tą metodę, warunkiem zatrzymania algorytmu musi być ilość iteracji.
\begin{gather}
    wt = wt_{min} + (wt_{max} - wt_{min})\frac{(t_{max} - t)}{t_{max}},\\
    c_1 = (c_{1f} - c_{1i})\frac{t}{t_{max}} + c_{1i},\\
    c_2 = (c_{2f} - c_{2i})\frac{t}{t_{max}} + c_{2i}
\end{gather}

\subsection{Propozycja zmiany równania prędkości}

\section{Analiza wpływu parametrów przyspieszenia na charakterystykę przebiegu algorytmu}
Wszystkie eksperymenty w niniejszej sekcji wykonane zostały dla 30 cząstek dla funkcji sfery:
\begin{gather}
    f_R = \sum_{i=1}^n x_i^2,\\
    gdzie\; x\in(-100, 100),\; n=20
\end{gather}

% \begin{equation}
%     f_{R} = \sum_{i=1}^n (x_i^2 -10cos(2\pi x_i)),
%     \label{eq:rastrigin}
% \end{equation}\

\subsection{Współczynniki stałe}
W tej sekcji warunkiem zatrzymania algorytmu jest precyzja $|f_R(x) - f_R(x_min)| \leq 0.0001$ lub przekroczenie 10 000 iteracji.
\begin{figure}[H]
    \centering
    \begin{subfigure}{0.49\textwidth}
        \centering
        \begin{adjustbox}{width=\linewidth}
        \begin{tikzpicture}
            \begin{axis}[grid=both,
                        grid style={solid,gray!30!white},
                        axis lines=middle,
                        xlabel={Iteracja},
                        ylabel={Błąd},
                        x label style={at={(axis description cs:0.5,-0.1)},anchor=north},
                        y label style={at={(axis description cs:-0.2,0.5)},anchor=south},
                        ]
            \addplot table [
                mark = none,
                blue,
                x=Step, y=Value,
                col sep=comma] {plots/ex1.csv};
            \end{axis}
        \end{tikzpicture}
        \end{adjustbox}
        \caption{$c1,c2=1$, 10 000 iteracji, końcowa wartość:~3032.7788}
    \end{subfigure}
    \begin{subfigure}{0.49\textwidth}
        \centering
        \begin{adjustbox}{width=\linewidth}
        \begin{tikzpicture}
            \begin{axis}[grid=both,
                        grid style={solid,gray!30!white},
                        axis lines=middle,
                        xlabel={Iteracja},
                        ylabel={Błąd},
                        x label style={at={(axis description cs:0.5,-0.1)},anchor=north},
                        y label style={at={(axis description cs:-0.2,0.5)},anchor=south},
                        ]
            \addplot table [
                mark = none,
                blue,
                x=Step, y=Value,
                col sep=comma] {plots/ex2.csv};
            \end{axis}
        \end{tikzpicture}
        \end{adjustbox}
        \caption{$c1,c2=2$, 243 iteracje, końcowa wartość:~0.00009}
    \end{subfigure}
    \caption{Wykresy przebiegu algorytmu dla $w=0.3\;$.}
\end{figure}
\begin{figure}[H]
    \centering
    \begin{subfigure}{0.49\textwidth}
        \centering
        \begin{adjustbox}{width=\linewidth}
        \begin{tikzpicture}
            \begin{axis}[grid=both,
                        grid style={solid,gray!30!white},
                        axis lines=middle,
                        xlabel={Iteracja},
                        ylabel={Błąd},
                        x label style={at={(axis description cs:0.5,-0.1)},anchor=north},
                        y label style={at={(axis description cs:-0.2,0.5)},anchor=south},
                        ]
            \addplot table [
                mark = none,
                blue,
                x=Step, y=Value,
                col sep=comma] {plots/ex3.csv};
            \end{axis}
        \end{tikzpicture}
        \end{adjustbox}
        \caption{$c1,c2=1$, 10 000 iteracji, końcowa wartość:~171.53}
    \end{subfigure}
    \begin{subfigure}{0.49\textwidth}
        \centering
        \begin{adjustbox}{width=\linewidth}
        \begin{tikzpicture}
            \begin{axis}[grid=both,
                        grid style={solid,gray!30!white},
                        axis lines=middle,
                        xlabel={Iteracja},
                        ylabel={Błąd},
                        x label style={at={(axis description cs:0.5,-0.1)},anchor=north},
                        y label style={at={(axis description cs:-0.2,0.5)},anchor=south},
                        ]
            \addplot table [
                mark = none,
                blue,
                x=Step, y=Value,
                col sep=comma] {plots/ex4.csv};
            \end{axis}
        \end{tikzpicture}
        \end{adjustbox}
        \caption{$c1,c2=2$, 695 iteracji, końcowa wartość: 0.00009}
    \end{subfigure}
    \caption{Wykresy przebiegu algorytmu dla $w=0.6\;$.}
\end{figure}

\subsection{Współczynniki zmienne (TVAC)}
W tej sekcji algorytm wykonuje się przez 500 iteracji niezależnie od wyników. Wartości początkowe i końcowe dla współczynników to:
\begin{itemize}
    \item $w_{init} = 0.9,\; w_{final} = 0.4$
    \item $c_{1init} = 2,\; c_{1final} = 1$
    \item $c_{2init} = 1,\; c_{2final} = 2$
\end{itemize}
\begin{figure}[H]
    \centering
    \begin{adjustbox}{width=0.6\linewidth}
    \begin{tikzpicture}
        \begin{axis}[grid=both,
                    grid style={solid,gray!30!white},
                    axis lines=middle,
                    xlabel={Iteracja},
                    ylabel={Błąd},
                    x label style={at={(axis description cs:0.5,-0.1)},anchor=north},
                    y label style={at={(axis description cs:-0.2,0.5)},anchor=south},
                    ]
        \addplot table [
            mark = none,
            blue,
            x=Step, y=Value,
            col sep=comma] {plots/ex5.csv};
        \end{axis}
    \end{tikzpicture}
\end{adjustbox}
    \caption{Wykresy przebiegu algorytmu dla zmiennych wartości współczynników przyspieszenia. 1000 iteracji, końcowa wartość: 0.000008}
\end{figure}

Na podstawie wyników można zauważyć, że najlepsze wyniki dla współczynników stałych osiągane są, gdy parametr $w$ jest mniejszy, a parametry $c_1$ oraz $c_2$ są większe. Oznacza to, że algorytm jest efektywniejszy, gdy prędkości kognitywne i społeczne są ważniejsze niż prędkość własna. w związku z tym, cząstki bardziej podążają w kierunku najlepszych osiągniętych wyników a mniej błądzą w przypadkowych kierunkach.

Zmienne współczynniki przyspieszenia zapewniają bardziej metodyczne przeszukiwanie. Jak widać na wykresach, największe różnice w wartościach widoczne są na początku. Po kilku iteracjach wykres się wypłaszcza i osiągnięcie mniejszych wyników zajmuje coraz więcej czasu. Zmienne przyspieszenie wzmacnia ten efekt pozwalając na większe skoki na początku i coraz mniejsze im bardziej algorytm zbliża się do minimum.

\section{Wyniki przebiegu algorytmu dla różnych funkcji}
W nawiązaniu do poprzedniej sekcji, przy przeprowadzaniu poniższych eksperymentów zdecydowałem się użyć najlepszej według mnie wersji, tj. zmiennych współczynników przyspieszenia. W związku z tym wyborem, każdy eksperyment trwał 300 iteracji.
Każda funkcja przetestowana została na zasugerowanym w instrukcji przedziale.
\begin{figure}[H]
    \centering
    \begin{subfigure}{0.32\textwidth}
        \centering
        \begin{adjustbox}{width=\linewidth}
        \begin{tikzpicture}
            \begin{axis}[grid=both,
                        grid style={solid,gray!30!white},
                        axis lines=middle,
                        xlabel={Iteracja},
                        ylabel={Błąd},
                        x label style={at={(axis description cs:0.5,-0.1)},anchor=north},
                        y label style={at={(axis description cs:-0.2,0.5)},anchor=south},
                        ]
            \addplot table [
                mark = none,
                blue,
                x=Step, y=Value,
                col sep=comma] {plots/ex6.csv};
            \end{axis}
        \end{tikzpicture}
        \end{adjustbox}
        \caption{Populacja 20, końcowy wynik: $1.28$}
    \end{subfigure}
    \begin{subfigure}{0.32\textwidth}
        \centering
        \begin{adjustbox}{width=\linewidth}
        \begin{tikzpicture}
            \begin{axis}[grid=both,
                        grid style={solid,gray!30!white},
                        axis lines=middle,
                        xlabel={Iteracja},
                        ylabel={Błąd},
                        x label style={at={(axis description cs:0.5,-0.1)},anchor=north},
                        y label style={at={(axis description cs:-0.2,0.5)},anchor=south},
                        ]
            \addplot table [
                mark = none,
                blue,
                x=Step, y=Value,
                col sep=comma] {plots/ex7.csv};
            \end{axis}
        \end{tikzpicture}
        \end{adjustbox}
        \caption{Populacja 50, końcowy wynik: $8.89\cdot 10^{-7}$}
    \end{subfigure}
    \begin{subfigure}{0.32\textwidth}
        \centering
        \begin{adjustbox}{width=\linewidth}
        \begin{tikzpicture}
            \begin{axis}[grid=both,
                        grid style={solid,gray!30!white},
                        axis lines=middle,
                        xlabel={Iteracja},
                        ylabel={Błąd},
                        x label style={at={(axis description cs:0.5,-0.1)},anchor=north},
                        y label style={at={(axis description cs:-0.2,0.5)},anchor=south},
                        ]
            \addplot table [
                mark = none,
                blue,
                x=Step, y=Value,
                col sep=comma] {plots/ex8.csv};
            \end{axis}
        \end{tikzpicture}
        \end{adjustbox}
        \caption{Populacja 100, końcowy wynik: $8.53\cdot 10^{-11}$}
    \end{subfigure}
    \caption{Wykresy przebiegu algorytmu dla funkcji sfery dla $x\in(-100,100)$ dla 20 wymiarów.}
\end{figure}

\begin{figure}[H]
    \centering
    \begin{subfigure}{0.32\textwidth}
        \centering
        \begin{adjustbox}{width=\linewidth}
        \begin{tikzpicture}
            \begin{axis}[grid=both,
                        grid style={solid,gray!30!white},
                        axis lines=middle,
                        xlabel={Iteracja},
                        ylabel={Błąd},
                        x label style={at={(axis description cs:0.5,-0.1)},anchor=north},
                        y label style={at={(axis description cs:-0.2,0.5)},anchor=south},
                        ]
            \addplot table [
                mark = none,
                blue,
                x=Step, y=Value,
                col sep=comma] {plots/ex9.csv};
            \end{axis}
        \end{tikzpicture}
        \end{adjustbox}
        \caption{Populacja 20, końcowy wynik: $17.87$}
    \end{subfigure}
    \begin{subfigure}{0.32\textwidth}
        \centering
        \begin{adjustbox}{width=\linewidth}
        \begin{tikzpicture}
            \begin{axis}[grid=both,
                        grid style={solid,gray!30!white},
                        axis lines=middle,
                        xlabel={Iteracja},
                        ylabel={Błąd},
                        x label style={at={(axis description cs:0.5,-0.1)},anchor=north},
                        y label style={at={(axis description cs:-0.2,0.5)},anchor=south},
                        ]
            \addplot table [
                mark = none,
                blue,
                x=Step, y=Value,
                col sep=comma] {plots/ex10.csv};
            \end{axis}
        \end{tikzpicture}
        \end{adjustbox}
        \caption{Populacja 50, końcowy wynik: $16.25$}
    \end{subfigure}
    \begin{subfigure}{0.32\textwidth}
        \centering
        \begin{adjustbox}{width=\linewidth}
        \begin{tikzpicture}
            \begin{axis}[grid=both,
                        grid style={solid,gray!30!white},
                        axis lines=middle,
                        xlabel={Iteracja},
                        ylabel={Błąd},
                        x label style={at={(axis description cs:0.5,-0.1)},anchor=north},
                        y label style={at={(axis description cs:-0.2,0.5)},anchor=south},
                        ]
            \addplot table [
                mark = none,
                blue,
                x=Step, y=Value,
                col sep=comma] {plots/ex11.csv};
            \end{axis}
        \end{tikzpicture}
        \end{adjustbox}
        \caption{Populacja 100, końcowy wynik: $14.01$}
    \end{subfigure}
    \caption{Wykresy przebiegu algorytmu dla funkcji Rosenbrocka dla $x\in(-2.048,2.048)$.}
\end{figure}

\begin{figure}[H]
    \centering
    \begin{subfigure}{0.32\textwidth}
        \centering
        \begin{adjustbox}{width=\linewidth}
        \begin{tikzpicture}
            \begin{axis}[grid=both,
                        grid style={solid,gray!30!white},
                        axis lines=middle,
                        xlabel={Iteracja},
                        ylabel={Błąd},
                        x label style={at={(axis description cs:0.5,-0.1)},anchor=north},
                        y label style={at={(axis description cs:-0.2,0.5)},anchor=south},
                        ]
            \addplot table [
                mark = none,
                blue,
                x=Step, y=Value,
                col sep=comma] {plots/ex12.csv};
            \end{axis}
        \end{tikzpicture}
        \end{adjustbox}
        \caption{Populacja 20, końcowy wynik: $43.81$}
    \end{subfigure}
    \begin{subfigure}{0.32\textwidth}
        \centering
        \begin{adjustbox}{width=\linewidth}
        \begin{tikzpicture}
            \begin{axis}[grid=both,
                        grid style={solid,gray!30!white},
                        axis lines=middle,
                        xlabel={Iteracja},
                        ylabel={Błąd},
                        x label style={at={(axis description cs:0.5,-0.1)},anchor=north},
                        y label style={at={(axis description cs:-0.2,0.5)},anchor=south},
                        ]
            \addplot table [
                mark = none,
                blue,
                x=Step, y=Value,
                col sep=comma] {plots/ex13.csv};
            \end{axis}
        \end{tikzpicture}
        \end{adjustbox}
        \caption{Populacja 50, końcowy wynik: $42.78$}
    \end{subfigure}
    \begin{subfigure}{0.32\textwidth}
        \centering
        \begin{adjustbox}{width=\linewidth}
        \begin{tikzpicture}
            \begin{axis}[grid=both,
                        grid style={solid,gray!30!white},
                        axis lines=middle,
                        xlabel={Iteracja},
                        ylabel={Błąd},
                        x label style={at={(axis description cs:0.5,-0.1)},anchor=north},
                        y label style={at={(axis description cs:-0.2,0.5)},anchor=south},
                        ]
            \addplot table [
                mark = none,
                blue,
                x=Step, y=Value,
                col sep=comma] {plots/ex14.csv};
            \end{axis}
        \end{tikzpicture}
        \end{adjustbox}
        \caption{Populacja 100, końcowy wynik: $40.79$}
    \end{subfigure}
    \caption{Wykresy przebiegu algorytmu dla funkcji Rastrigina dla $x\in(-5.12,5.12)$.}
\end{figure}

\begin{figure}[H]
    \centering
    \begin{subfigure}{0.32\textwidth}
        \centering
        \begin{adjustbox}{width=\linewidth}
        \begin{tikzpicture}
            \begin{axis}[grid=both,
                        grid style={solid,gray!30!white},
                        axis lines=middle,
                        xlabel={Iteracja},
                        ylabel={Błąd},
                        x label style={at={(axis description cs:0.5,-0.1)},anchor=north},
                        y label style={at={(axis description cs:-0.2,0.5)},anchor=south},
                        ]
            \addplot table [
                mark = none,
                blue,
                x=Step, y=Value,
                col sep=comma] {plots/ex15.csv};
            \end{axis}
        \end{tikzpicture}
        \end{adjustbox}
        \caption{Populacja 20, końcowy wynik: $2.72\cdot 10^{-3}$}
    \end{subfigure}
    \begin{subfigure}{0.32\textwidth}
        \centering
        \begin{adjustbox}{width=\linewidth}
        \begin{tikzpicture}
            \begin{axis}[grid=both,
                        grid style={solid,gray!30!white},
                        axis lines=middle,
                        xlabel={Iteracja},
                        ylabel={Błąd},
                        x label style={at={(axis description cs:0.5,-0.1)},anchor=north},
                        y label style={at={(axis description cs:-0.2,0.5)},anchor=south},
                        ]
            \addplot table [
                mark = none,
                blue,
                x=Step, y=Value,
                col sep=comma] {plots/ex16.csv};
            \end{axis}
        \end{tikzpicture}
        \end{adjustbox}
        \caption{Populacja 50, końcowy wynik: $1.74\cdot 10^{-4}$}
    \end{subfigure}
    \begin{subfigure}{0.32\textwidth}
        \centering
        \begin{adjustbox}{width=\linewidth}
        \begin{tikzpicture}
            \begin{axis}[grid=both,
                        grid style={solid,gray!30!white},
                        axis lines=middle,
                        xlabel={Iteracja},
                        ylabel={Błąd},
                        x label style={at={(axis description cs:0.5,-0.1)},anchor=north},
                        y label style={at={(axis description cs:-0.2,0.5)},anchor=south},
                        ]
            \addplot table [
                mark = none,
                blue,
                x=Step, y=Value,
                col sep=comma] {plots/ex17.csv};
            \end{axis}
        \end{tikzpicture}
        \end{adjustbox}
        \caption{Populacja 100, końcowy wynik: $8.92\cdot 10^{-10}$}
    \end{subfigure}
    \caption{Wykresy przebiegu algorytmu dla funkcji Browna dla $x\in(-4,4)$.}
\end{figure}

\begin{figure}[H]
    \centering
    \begin{subfigure}{0.32\textwidth}
        \centering
        \begin{adjustbox}{width=\linewidth}
        \begin{tikzpicture}
            \begin{axis}[grid=both,
                        grid style={solid,gray!30!white},
                        axis lines=middle,
                        xlabel={Iteracja},
                        ylabel={Błąd},
                        x label style={at={(axis description cs:0.5,-0.1)},anchor=north},
                        y label style={at={(axis description cs:-0.2,0.5)},anchor=south},
                        ]
            \addplot table [
                mark = none,
                blue,
                x=Step, y=Value,
                col sep=comma] {plots/ex18.csv};
            \end{axis}
        \end{tikzpicture}
        \end{adjustbox}
        \caption{Populacja 20, końcowy wynik: $0$}
    \end{subfigure}
    \begin{subfigure}{0.32\textwidth}
        \centering
        \begin{adjustbox}{width=\linewidth}
        \begin{tikzpicture}
            \begin{axis}[grid=both,
                        grid style={solid,gray!30!white},
                        axis lines=middle,
                        xlabel={Iteracja},
                        ylabel={Błąd},
                        x label style={at={(axis description cs:0.5,-0.1)},anchor=north},
                        y label style={at={(axis description cs:-0.2,0.5)},anchor=south},
                        ]
            \addplot table [
                mark = none,
                blue,
                x=Step, y=Value,
                col sep=comma] {plots/ex19.csv};
            \end{axis}
        \end{tikzpicture}
        \end{adjustbox}
        \caption{Populacja 50, końcowy wynik: $0$}
    \end{subfigure}
    \begin{subfigure}{0.32\textwidth}
        \centering
        \begin{adjustbox}{width=\linewidth}
        \begin{tikzpicture}
            \begin{axis}[grid=both,
                        grid style={solid,gray!30!white},
                        axis lines=middle,
                        xlabel={Iteracja},
                        ylabel={Błąd},
                        x label style={at={(axis description cs:0.5,-0.1)},anchor=north},
                        y label style={at={(axis description cs:-0.2,0.5)},anchor=south},
                        ]
            \addplot table [
                mark = none,
                blue,
                x=Step, y=Value,
                col sep=comma] {plots/ex20.csv};
            \end{axis}
        \end{tikzpicture}
        \end{adjustbox}
        \caption{Populacja 100, końcowy wynik: $0$}
    \end{subfigure}
    \caption{Wykresy przebiegu algorytmu dla funkcji Corany dla $x\in(-1000,1000)$ dla 4 wymiarów.}
\end{figure}

\section{Porównanie wyników algorytmu PSO i algorytmu GA}
Porównanie zostało przeprowadzone dla wyników GA z pracy \cite{comparison} oraz parametrów PSO z tego samego dokumentu dla maksymalnej wiarygodności otrzymanych wyników. W ostatnim rzędzie przedstawione są wartości dla algorytmu PSO ze zmiennymi wartościami współczynników przyspieszenia.
\begin{table}[H]
    \centering
    \begin{tabular}{|c|c|c|}
        \hline
        &\textbf{Sfera}&\textbf{Rastrigin}\\
        \hline
        \textbf{PSO}&$6.22$&$285.09$\\
        \hline
        \textbf{GA}&$0.0008$&$0.05$\\
        \hline
        \textbf{PSO TVAC}&$1.75\cdot 10^{-11}$&$30.84$\\
    \end{tabular}
    \caption{Porównanie wyników algorytmu PSO i GA.}
\end{table}

\bibliographystyle{plain}
\bibliography{references.bib}
\end{document}